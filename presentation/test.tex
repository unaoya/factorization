\documentclass{beamer}
\usepackage{moreverb}
\input{../../tex/theorems}
\input{../../tex/symbols}
\usepackage{url}

\title{Egisonによる因数分解}
\author{梅崎直也@unaoya}
\date{Egison Workshop 2018, 2018年11月23日}
\institute{株式会社すうがくぶんか}

\begin{document}

\begin{frame}
\maketitle
\end{frame}

\begin{frame}{自己紹介}
\begin{itemize}
\item 梅崎直也@unaoya
\item 東大数理科学研究科出身
\item 数学を教える仕事
\item 昨年末からEgisonのバイト
\end{itemize}
\end{frame} 

\begin{frame}{$\F_p[x]$での因数分解}
例えば$p=3$では

\begin{align*}
x^3+x^2+x+1&=(x^2+1)(x+1)\\
x^2+2&=x^2+3x+2\\&=(x+1)(x+2)\\
x^3+1&=x^3+3x^2+3x+1\\&=(x+1)^3
\end{align*}
\end{frame}

\begin{frame}{$\F_p[x]$での演算}
多項式の四則、ベキ乗、割り算、gcd、微分などを係数$\mod p$する。
\verbatimtabinput{operation.egi}
\end{frame}

\begin{frame}{因数分解アルゴリズムの実装}
次の三段階で因数分解を行うアルゴリズム

参考:Wikipedia, Factorization of polynomials over finite fields

\begin{enumerate}
\item $f(x)$を重複度ごとに分解、Square-free factorization
\item 既約因子の次数ごとに分解、Distict-degree factorization
\item 次数成分ごとに既約分解、Cantor-Zassenhaus algorithm
\end{enumerate}
\end{frame}

\begin{frame}{Square-free factorization}

\begin{align*}
f(x)=f_1(x)f_2(x)^2f_3(x)^3\cdots f_n(x)^n
\end{align*}
と各因子$f_i(x)$は重複因子を持たず、それぞれ互いに素であるように分解


\begin{align*}
f(x)&=x^{11}+2x^9+2x^8+x^6+x^5+2x^3+2x^2+1 \in \F_3[x]\\
&=(x+1)(x^2+1)^3(x+2)^4
\end{align*}
\end{frame}

\begin{frame}{SFF実装}

$f'$と$f$の共通約数が重複因子であることを用いる。
ただし標数$p>0$のときは$(x^p)'=0$なので注意が必要。

\verbatimtabinput{sff.egi}

\end{frame}

\begin{frame}{SFF実装}
\verbatimtabinput{sff2.egi}
\end{frame}


\begin{frame}{Distict-degree factorization}

\begin{align*}
f(x)=f_1(x)f_2(X)\cdots f_d(x)
\end{align*}
で$f_i(x)$の既約因子が$i$次式であるように分解する。

\begin{align*}
x^4+2=(x^2+2)(x^2+1)
\end{align*}
\end{frame}

\begin{frame}{DDF実装}
$\F_{q^i}$の乗法群が位数$q^i-1$の有限群であることから、$x^{q^i}-x$との公約数を取ればよい。

\verbatimtabinput{ddf.egi}
\end{frame}

\begin{frame}{Cantor-Zassenhaus}
重複因子を持たず、既約成分が全て$d$次である多項式を因数分解する。
\verbatimtabinput{cz.egi}

\end{frame}

\begin{frame}{今後の課題}
\begin{enumerate}
\item 有理数係数での因数分解(Hensel持ち上げ)
\item 代数的数の扱い(書き換え規則でできそう)
\item 積分
\item いちいち$\mod p$での演算を実装し直すのが面倒
\end{enumerate}
\end{frame}

\begin{frame}{参考文献}
\begin{enumerate}
\item Wikipedia, Factorization of polynomials over finite fields,
\item \url{https://github.com/unaoya/factorization}
\end{enumerate}
\end{frame}

\end{document}